\section{Examples from real life}

% Describe several examples how well-known companies apply real-time processing to the real data.
% What algorithms do they maybe use, what topology or schema of computations in general.
% 
% The real
% Many companies as well as 
% 
% Twitter uses real-time processing immensely \cite{Toshniwal2014} \cite{Boykin2013}.

Nowadays real-time processing founds its uses in many spheres of life.
It can be found in such fields as stock market trading, market data management, pricing and analytics, e-commerce, risk management, fraud detection, network monitoring and many others. 
Let us describe in more detail several examples.

\mnote{electronic trading}
Currently floor-based trading in stock market field is replaced by electronic trading.
This approach brings together buyers and sellers using an electronic trading platform.
It gives a superior advantage against human-based trading because it can provide the fastest reaction and high-speed calculations.

But this property also puts strict requirements on the system.
For instance, we can take an official statistics from Options Price Reporting Authority (OPRA) that aggregates from options exchanges all the quotes and trades.   
It recorded a peak of 31 million messages per second in January 2014 with a prediction that this number will constantly grow.
Furthermore, electronic trading systems puts high demands on latency.
Latency in this case has a direct impact on profit.
System that is able to produce the most current result maximizes the profit. 

A variety of electronic trading platforms are created, that meet the mentioned requirements.
Some existing examples include CME Globex, Goldman Sachs, CQG, eExchange. 
These platforms place orders over a network for stocks, currencies, bonds, etc.
Typically an electronic trading platform provides a stream of live market prices, using dedicated real-time processing algorithms.
The detailed observation of the inner structure of such systems requires in-depth study and is out of scope of our work.

\mnote{military}
Another example of application the real-time processing is military technologies.
Technological progress allows to change the fighting methods.
Soldiers can be supplied with vital-signs monitors.
GPS sensors can be placed into military vehicles and even can be given to every soldier.
This allows to monitor the overall situation of military operation in real time.
Consequently it is required to use specific real-time processing algorithms to obtain the needed results and perform necessary calculations.

Sensor-based monitoring also can be used in other, non-military domains.
For example, automobiles and public transport vehicles, supplied with GPS sensors, can report about the situation on the roads to avoid traffic problems. 
Sensors, placed inside industrial pipes or premises with a potential danger of disaster, is another example.
Data obtained from such sensors helps to monitor the system and detect in proper time the anomaly to avoid a catastrophe.
 
The examples mentioned above represents a small part of real-time processing applications.
All of these systems require a dedicated approach and need specific algorithms that allow to handle real-time data most efficiently.
Some of these algorithms we describe in the following section.

