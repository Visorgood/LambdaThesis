\chapter{Introduction}
\label{chap:introduction}

Both psychology and psychiatry heavily rely on observing humans.
They fundamentally assume that a person's mental state affects his interactions with the environment.
The other way around it is possible to derive a person's mental state by observing his actions.
If a driver maneuvers his car more aggressively than usual, we can safely assume that he is stressed.
It is however inherently hard to gather such data.

Traditional observation methods are too expensive and often inapplicable.
One approach is to let users collect data about themselves, e.g.\ by using a diary.
Another one is to keep them in a controlled environment that makes it possible for external parties to note down their actions.
A third approach is shadowing, where a follower observes a user's every move and collects this data for him.
However all those methods inherently restrict the user and are biased towards the observer.
They are also expensive and very time consuming which makes them infeasible for a broader audience.
Recent technological developments allow the collection of data in a far less intrusive way.

Smartphones are a quantity rich source of data.
In contrast to  mobile sensors like wristbands, which e.g.\ measure the heart rate, they are already an everyday companion for many users.
They take them wherever they go and use them throughout the day, mostly performing numerous short interactions.
For many people it is the first and the last device they use during their waking hours.
They possess a large number of different sensors which provide analyzable data.
Additionally, they capture a big part of people's social interactions.
This makes them especially suited for this use case.



% Every process conducts potential data.
% In the digital age more and more processes in the real world actually produce data.
% Back in the days all conversations were held face to face.
% Nowadays people often send short messages to each other via mobile devices.
% Asking for the way becomes unnecessary when everybody has a navigation system built into their car.
% This also captures the location of the person at every point in time.
% These are only two exemplary situations where usable data occurs which remains unused yet.
% 
% The central thesis is that emotions and inner processes manifest in physical interactions of people with their devices.
% An upset person will drive his car more aggressively than a balanced one.
% The other way around we are able to extract a person's mood by observing his behavior.
% Smartphones are especially suited to observe such interactions.
% Users take them wherever they go and use them throughout the day, mostly performing numerous short interactions.
% For many people it is the first and the last device they use during their waking hours.
% They possess a number of different sensors, which provide analyzable data.
% 
% Actually collecting, storing and investigating this data can be subsumed with the term Big Data.
% It describes collecting as much data as possible without any primary target.
% The idea is to answer some question in the future by having sufficient data to incorporate.
% This leads to the idea of the Menthal (menthal health analytics) project.

Menthal (mental health analytics) is a research project on the intersection between psychology and mobile computing.
At its core it contains a mobile software that logs interactions, locations and app usages of users.
This application is used by several subprojects which all focus on human behavior.
%Several subprojects incorporate this software to investigate human behavior.
OpenMenthal is a smartphone app for Self Monitoring where people analyze themselves on their own.
They are supported by automatic data collection and aggregation.
The users can review their data as graphs of time series.
MenthalDepressed tries to capture the mood level of a person by observing changes in their interactions with other people.
If someone for instance starts talking to fewer and fewer people, it might be an indicator of some mental issue.
Both deal with the analysis of potentially large data sets.
This poses new challenges for storing, managing and processing this data.

In this thesis we present a structured approach for finding a suitable data storage technology to cope with the challenges of our particular use case.
These are the Big Data scenario, the high frequency of inserts and the ability to handle continuous queries.
% More precisely we will deal with large amounts of data which will not fit on a single machine.
The data is heterogeneous because the type and characteristics of data sources may change over time.
We can think of including new sensors or adapting to different mobile phones.
We expect high rates of new incoming data which a system has to handle.
%The execution of continuous queries has to meet strict time constraints to be able to handle the incoming stream of data.
The arrival of new data triggers aggregations which we thus have to perform continuously in a given time frame.
In order to fulfill these requirements in the limited time frame of our thesis we have to exhaust all given assumptions.
For example, we do not join the data of different users.
This level of optimization requires revisiting every single aspect of traditional database system architectures.

This thesis consists of twelve chapters which can be divided into three parts.
The first part builds the foundation by introducing the mobile data scenario and data management in general.
Chapter 2 introduces the fundamentals of capturing data on mobile devices.
It also discusses the common characteristics and particularities of time series data in this context.
Chapter 3 presents the basic aspects of data storage.
It addresses the question of how to store and retrieve data efficiently.
Chapter 4 compares the most common database models, based on the previously introduced concepts.
It discusses traditional approaches like the relational model as well as the more novel NoSQL systems.
Chapter 5 turns to the real world and provides an overview of the database system market.
In particular, it outlines four archetypical systems we employ throughout the rest of this thesis.
These are the EventStore from Marburg University, the column-family store HBase, the relational database system PostgreSQL and the in-memory key-value store Redis.

The second part deals with the requirements of our particular scenario and proposes a suitable architecture.
The sixth chapter takes a closer look at the two main use cases within the Menthal project, namely Quantified Self (Self Tracking) and illness detection.
It declares the focus on Quantified Self and extracts functional and nonfunctional requirements from the specified use case description.
On this foundation Chapter 7 proposes an architecture for a Menthal backend.
It describes the different components and their basic responsibilities.
It states problems which arise and sketches solutions.
However it does not provide a full implementation description.

The third part deals with the most compelling question of what database technology fits the best within this architecture.
To this end Chapter 8 describes a new benchmark for testing different database systems.
In contrast to existing benchmarks it focuses on the data and query load we expect in the scope of the project.
The chapter begins by describing an algorithm for artificial data generation based on empirical studies.
Furthermore, it defines queries to review certain aspects of the investigated systems.
Chapter 9 deals with the implementation of the benchmark and the adaptation of the different database systems.
It presents a detailed description of the data generation and specifies for each system a way of storing and querying this data.
The tenth chapter presents the results of an experiment based on the benchmark.
It interprets the observations and points out the differences in the performance of the systems.
Chapter 11 treats the non-performance aspect of the database systems.
It presents our experiences during the implementation phase of the benchmark and gives an impression of how well the systems support the development process.
Finally, Chapter 12 concludes the thesis by summarizing our work and providing an outlook on future research questions.






% 
% Bei jedem Prozess fallen potentiell Daten an.
% Im digitalen Zeitalter ist dies immer h�ufiger tats�chlich der Fall.
% Wo fr�her eine Gespr�ch face to face stattfand, wird heute eine SMS geschickt.
% Fr�her hat man nach dem WEeg gefragt oder Landkarten verwendet.
% Heute hat jedermann ein Navigationsger�t im Auto.
% Dieses erfasst zudem jederzeit den STandort der betreffenden Person.
% 
% Dies sind nur zwei Beispiele in denen verwertbare Daten anfallen, die bisher aber nicht genutzt werden.
% Hier setzt unsere zentrale These an, dass sich Gef�hle und innere Vorg�nge in den physischen Aktionen von Menschen mit Ger�ten wiederspiegeln.
% Ein aufgeregter Mensch wird aggresiver im STra�enverkehr vorgehen, als ein ausgeglichener.
% Andersherum lassen sich aus den Interaktionen von Menschen mit Maschinen R�ckschl�sse auf ihren Gem�tszustand ziehen.
% 
% 
% 
% 
% three approaches
%   - self tracking
%     - people can easily collect data about themselves
%   - cell phone abuse
%   - smartphones
%     - people use smartphones all day errday
%     - there are lots of information to gather on the devices
%     - Big Data approach
%       - collect as much data as possible
%       - make use of this data
%       - question makein
%         - self monitoring/quantified self
%         - analysis
%           - 
% 
% 
% People collect data about themselves for ages.
% Typically sportsmen track their training progress over large periods of time.
% Health related issues on the other hand are traditionally measured by doctors.
% In recent years
% 
% Introduction\ldots
% 
% - Menthal project
%   - target
%     - stress detection
%       - mistakes in texts in comparison to normal
%     - depressions
%       - motion radius
%       - number of people talk to
%     - misuse of mobile telecommunication devices
%   - 24/7 observation (not possible in hospital/psychiatry)
%   - collect much data without knowing relevance
%     -> Big Data approach
%   - anaonymization/preserve privacy
%     - ethical concerns
%     - confidentiality
%     - provide content to other scientists
%   - collect events
%   - visualize collected data/aggregation
%     - self observation
%   - stress detection algorithms
%   - HCI
%   - quantitative analysis
%   - early detection burnout
%   - alzheimers
%   - serious games
%     - ``enforce'' ``good'' behavior
%   - cooperation
%     - study of 150 participants
%     - psychologists
%     - correlations of personality traits and genetic markers
%   - OpenMenthal
%     - Smartphone app for publicity (android)
%   - MenthalDepressed
%     - monitor depression
%   - MenthalSkilled
%     - cognitive games in customized lockscreens
%   - MenthalDopa
%     - parkinsons
%     
%  - Structure of thesis
%    - Mobile sensor data characteristics
%      - time series
%      - mobile devices sensors (GPS, \ldots)
%    - foundations of data base systems
%    - database models
%    - database market overview/taxonomy/classification
%    - benchmark
%      - what is a benchmark
%      - related work
%      - our concept
%      - our implementation
%      - our results
%    - Concepts for DataStore 1