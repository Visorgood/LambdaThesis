\chapter{Real time processing in the Big Data context [VI] [14 pages]}
\label{chap:real_time_processing}

[1 page]

Common paragraph

Web-companies encounter nowdays necessity to process data streams in the real-time fashion.
These streams produce often vast amount of data in a short period of time.
There are plenty of examples: Google's Zeitgeist system that tracks trends in search queries (some more examples\ldots)

There are many algorithms for processing data streams.
They divide to categories depending on what type of query do they answer.
A few query types are: ``has an item already been observed?'', ``how many times has an item been observed?'', ``what is the number of unique items among observed?'', etc.

How all this can be used in the speed layer

\section{Examples from real life [5 pages]}

\section{Stream processing algorithms [5 pages]}

[1 page]

Stream processing algorithms, or sketch algorithms, allow to treat huge amounts of data coming from the stream in an efficient way.
They answer different types of queries, e.g. ``has particular item already been observed?'', ``how many times specific item has already arrived from the stream?'', ``what is the cardinality of the set of items so far came from the stream?'', etc.
Stream processing algorithms use usually small amount of memory to maintain their data structures.
At least in comparison with the amount of data that stream produces.

\subsection{Bloom filter [1 page]}
\subsection{Count-min sketch [1 page]}
\subsection{HyperLogLog algorithm [1 page]}
\subsection{Flajolet-Martin sketch [1 page]}

\section{Speed Layer (for real time processing) [3 pages]}