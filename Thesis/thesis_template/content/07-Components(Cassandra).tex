\subsection{Cassandra [VI]}
\label{subs:cassandra}

Cassandra is a distributed storage system \cite{Cassandra, Avinash2014, Hewitt2011}.
It is efficient and reliable, what is achieved via multinode topology and replication of data.
It keeps data on many commodity machines without having master node.
This leads to a purely distributed approach, where there is no a single point of a failure.
Cassandra is highly scalable, as long as any number of nodes can be added to the cluster at any time.
It provides CQL language, that allows writing and querying data in an SQL-style manner.

Cassandra's architecture is purely distributed.
It consists of a set of nodes, and all of them are equal in the sense of functionality.
Each node stores the part of data, what allows to extend capacity of the storage, basically, to any size.
Nodes exchange meta-information about the cluster's structure once per second.
Every node maintains a \textit{commit log} of writes.
Then it saves all written data to an in-memory data structure called \textit{memtable}, that functions as a cash.
When memtable is full, node flushes it to an SSTable file, partitioning and replicating data among the cluster.
Cassandra allows to connect to any node in the cluster, that works then as a coordinator.
It routs then requests of the client to nodes that have requested data.

\mnote{Main structures}
There are 5 main structures in Cassandra, that need to be defined.
\textit{Cluster} is a set of nodes that store the data.
\textit{Data center} is a subset of cluster's nodes that are grouped together.
They have a common configuration to achieve optimal replication and segregation properties.
Data still can be copied to several data centers, if replication factor is high.
\textit{Commit log} is a local storage on each node, where data is stored for durability, before being distributed across the cluster.
\textit{Table} is a logical place, where data is stored.
It is, essentially, a set of columns, that can be accessed by rows.
Each row has a primary key and a value, represented by a particular set of columns, that need not be the same for every row in the table.
\textit{SSTable} (sorted string table) is a file, where node periodically writes data, gathered to a memtable.
It is immutable, append-only, and exists for every Cassandra's table.

\mnote{Communication between nodes}
For communication between nodes Cassandra has a protocol called \textit{gossip}.
It allows to discover information about other nodes in the cluster, and to share information about state of the node.
According to gossip protocol every node communicates with three other nodes every second, exchanging meta-information about state, data, and other nodes.
Such network learns fast about its topology and state.

\mnote{Detection of failures and recovery}
Cassandra analyses state of nodes using gossip information to check whether node is up or down.
It tries whenever possible not to send any requests to nodes, that are out of service by any reason.
Particular node tries to understand, which other nodes are down directly, by communicating with them, as well as indirectly, receiving information about them from directly communicated nodes.
Node can fail due to many different reasons, e.g. hardware failures, software crashes, or network outages.
Cassandra does not remove the node from topology as soon as it seems to be down.
Instead, nodes try to re-establish communication with failed node.
It can be smart, because, for example, a network outgage can be recovered fast.
As long as node detected to be failed, other nodes start to replicate data bypassing it.
Administrator has to run a repairing tool, after node is up, to make it again consistent with the cluster.

\mnote{Data distribution}
Data distribution and replication are inherent properties of Cassandra.
Data is stored in tables, that consist of rows.
Row is an atomic peace of data in Cassandra.
It has a primary key, that defines its location in the cluster.
There are four aspects, that determine distribution and replication: virtual nodes, partitioners, replication strategies and snitches.
We discuss them shortly in details.

\mnote{Consistent hashing}
First, let us describe consistent hashing.
It divides possible values of partition keys into ranges.
Partition key is any column name of a row.
This allows to set for each node a subset of partition key values, that will be stored in this node.
Consistent hashing has an advantage, that it minimizes reorganization needs, when nodes are removed or added.
For a particular partition key, consistent hashing produces for each node an interval of hash-values.
Then for a new row, it is stored on the node, so that hash-value of key maps to the interval assigned to this node.

%Virtual nodes

\mnote{Data replication}
When new row is added, Cassandra replicates it across the cluster.
Copies of the row are called replicas.
Each replica is equal to others, and there is no any master or primary replica.
The number of replicas is called a \textit{replication factor}.
It defines on how many nodes replicas of a row will reside.
There are two strategies of a replication mechanism.
The first one is a \lstinline{SimpleStrategy}, that supposes to use only one data center for replication.
Another one is a \lstinline{NetworkTopologyStrategy}, that is used in most cases, and allows to replicate data across the whole cluster.

\mnote{Partitioners}
Now let us describe what is \textit{Partitioner}.
Partitioner defines how data is distributed across the cluster.
It computes a token from a partition key of a row.
This token is then used to choose nodes, where replicas of a row must be saved.
There are three partitioners in Cassandra: \lstinline{Murmur3Partitioner}, \lstinline{RandomPartitioner} and \lstinline{ByteOrderedPartitioner}.
\lstinline{Murmur3Partitioner} distributes data uniformly in the cluster using \lstinline{MurmurHash} hash function.
\lstinline{RandomPartitioner} distributed data uniformly using \lstinline{MD5} hash function.
\lstinline{ByteOrderedPartitioner} distributes data in a lexical order by key bytes.

\mnote{Snitches}
Snitches define mapping of nodes to data centers and racks.
Using snitches Cassandra understands how to route requests.
They provide topology information and allow to execute replications properly.
Snitch monitors historical information about efficiency of reading from different replicas of a row, and uses this information to choose the best one for a request.

When a client connects to Cassandra, it is able to communicate with any node. 
Requests are then redirected by a connected node to the one, that has a particular data to be returned to a client or to be written.
Connected node is called \textit{coordinator}.
Coordinator is, essentially, a proxy between a client and Cassandra store.

\mnote{CQL}
Cassandra has a query language \textit{CQL} (Cassandra query language).
It is an interface to communicate with Cassandra, to read and write data.
It is an SQL-style language, that allows to query data from Cassandra with different conditions.
As SQL it is based on a table representation of data.