\chapter{The Menthal Story}
\label{chap:menthal_story}

Nowadays mobile/smartphone addiction becomes a critical problem.
People constantly use the device, making calls, sending messages, surfing in the Internet or just playing.
More than a half of adult population of United States own a smartphone.
People are distracted by their phones even on personal meetings, parties and dates.
When a person leaves his phone at home, he feels nervious and irritated.
Some people suffer from a cellphone vibration syndrome, when one feels as if a cellphone is vibrating but infact it is not.
Sometimes they unlock the phone and check messages or social network pages unconsciously, just to make sure that nothing new has happened. 

And the crucial problem is that people are not able to detect this addiction.
Few of them can correctly specify the daily time expenditure on the device.
First, it is difficult to evaluate yourself from the outside.
Second, even if a person can correctly estimate the wasted time, it is morally unpleasant to confirm these figures.
Being aware of this statistics, it is possible to change your habits or even to go on a smartphone-free diet.
Thus, exterior assessment mechanism is needed to calculate how much time a user spend on his smartphone, what applications he uses and when.    

On the other hand, there is no official study how people use their phones.
For example, France introduses a rule to switch off work phones after 18:00 and till 9:00, to protect employees from official duties outside office hours.
"digital working time" would have to be measured 

1)app
-user want to be less addicted to his phone
2)study
-there is no official study how people use their phones
-the rule to switch phones at 18:00
-to make an interface more friendly
-to count how long do you use facebook
3)combining two
-how to force people use it? - application
4)technical architecture
client part + server part
5)resulting data (how many users, how manys recorded. how large data, how many .. per day, how many in total)


The rapid development of technologies, in particular smartphones, allows us to get a huge amount of data almost for free.
-mobile sensors + Internet
-the average duration of using a smartphone per day
-what information can be extracted from mobile sensor data

(The?) Menthal is an application for smartphones that gathers phone usage data and stores it on a server for further analysis, providing user with feetback. 
-nowadays Menthal has 50 000 active users
-we receive 1Tb data every month
-Menthal is still under development

The Menthal story started from a psychological experiment on 60 participants.
-the key idea
-methods: questionaries, brain scan
-data from phone sensors

This experiment developed into an independent smartphone application for mass usage. 
-it collects data, provides user with feedback and supplies psychologists with data
-it is settled on one machine with a PostgreSQL database
-it provides RESTful API and communicates via HTTP
-the data exchange is stateless
-the main problem is that it scales nicely for reads, but not for writes

The problem of scaling becomes urgent when Menthal is covered in mass media.
-newspapers (which?), interview on TV
-explosive increase of the number of users
-application is not able to write all incoming data into database, not to mention providing a feetback 

The architecture of the application is changed to meet the new requirements.
-mirrowed DB, caching - only improves reading, not writing
-sharding - makes things harder to implement and manage
-actually we deal with Big Data

The most suitable architectural solution is taking an advantage of recently developed Lambda Architecture.
-key properties
-why is it suitable
 
The data collected by Menthal can be used for various purposes.
-psychological research
-statistical calculations (what Apps are used most? how is it connected with user group?)
-services based on results on data analisys (offer applications to users)
-healthy person - depressed person - to find a desease at a proper time