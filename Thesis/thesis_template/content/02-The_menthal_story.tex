\chapter{The Menthal Story}
\label{chap:menthal_story}
Nowadays mobile/smartphone excessive usage becomes a critical problem.
People constantly use the device, making calls, sending messages, surfing in the Internet or just playing.
More than a half of adult population of United States own a smartphone.
People are distracted by their phones even on personal meetings, parties and dates.
When a person leaves his phone at home, he feels nervious and irritated.
Some people suffer from a cellphone vibration syndrome, when one feels as if a cellphone is vibrating but in fact it is not.
Sometimes they unlock the phone and check messages or social network pages unconsciously, just to make sure that nothing new has happened. 

The crucial problem is that people are not able to detect that they concentrate on their smartphones too much.
Few of them can correctly specify the daily time expenditure on the device.
First, it is difficult to evaluate yourself, take a detached view on your behavior.
For example, one can unlock his phone 45 times per day and use it for 2 minutes on average.
However, in the end of the day it sums up to 1,5 hours, what can surprise the user.
Second, even if a person can correctly estimate the wasted time, it is morally unpleasant to confirm these figures.
For instance, one plays a game on his smartphone instead of working on important task that has to be accomplished.
Finally, the task is not finished, the person realizes that the reason was that mobile game, but it is easier to say that there was not enough time to complete the task.  
Thus, exterior assessment mechanism is needed to calculate how much time a user spend on his smartphone, what applications he uses and when.    

At the same time, there is no proper study how do people use their phones.
Currently most of the researches in the field of human-smartphone interaction involve direct interaction with a user group, by means of questionnaires and interviews.
Recently special software applications for smartphones to keep track of user actions emerge.
However, they still require the interaction with users, showing dialog boxes and asking questions.
It introduses a certain bias into the results of the research, because a user gets distracted by this interference.
Only an application that is invisible for users can help to make a complete and thoroughtful research.

The results of human-smartphone study can be used in various fields.
For example, France introduses a rule to switch off work phones after 18:00 and till 9:00, to protect employees from official duties outside office hours.
To perform such introduction, it is important to investigate how do people of various professions use their work phone.
This research can help to detect the cases when it is actually necessary to block the phones.
In some cases maybe it is sufficient to block only particular functions, such as emails or text messages.

Moreover, it is essential to determine how people use their smartphones to make user-friendly application interfaces. 
An application interface has a big influence on user.
Developers can make it attractive and handy, that encourages users to use the application again even if there is no actual necessity.
If a person wants to decrease the time spent on the smartphone, a special enhanced interface can help.
For example, the phone can warn user when he spends too much time using the device during the day.  

As a result, there are two problems that have to be solved. 
On the one hand, there is a need to provide user with his smartphone use information.
On the other hand, it is necessary to supply researches with a tool that tracks smartphone usage.
Menthal, an application for smartphones, combines these two features and provides solutions for both these problems. 

Menthal is an application that gathers phone usage data and stores it on a server for further analysis, providing user with feedback.
% client part + server part

The Menthal story started from a psychological experiment on 60 participants.
% -the key idea
% -methods: questionaries, brain scan
% -data from phone sensors
% 
This experiment developed into an independent smartphone application for mass usage. 
% -it collects data, provides user with feedback and supplies psychologists with data
% -it is settled on one machine with a PostgreSQL database
% -it provides RESTful API and communicates via HTTP
% -the data exchange is stateless
% -the main problem is that it scales nicely for reads, but not for writes
% 
The problem of scaling becomes urgent when Menthal is covered in mass media.
% -newspapers (which?), interview on TV
% -explosive increase of the number of users
% -application is not able to write all incoming data into database, not to mention providing a feetback 
% 
The architecture of the application is changed to meet the new requirements.
% -mirrowed DB, caching - only improves reading, not writing
% -sharding - makes things harder to implement and manage
% -actually we deal with Big Data
% 
The most suitable architectural solution is taking an advantage of recently developed Lambda Architecture.
% -key properties
% -why is it suitable
%  
The data collected by Menthal can be used for various purposes.
% -psychological research
% -statistical calculations (what Apps are used most? how is it connected with user group?)
% -services based on results on data analisys (offer applications to users)
% -healthy person - depressed person - to find a desease at a proper time

resulting data (how many users, how manys recorded. how large data, how many .. per day, how many in total)
% -nowadays Menthal has 50 000 active users
% -we receive 1Tb data every month
% -Menthal is still under development