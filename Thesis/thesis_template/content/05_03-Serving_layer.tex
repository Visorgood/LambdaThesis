\section{Serving layer}

The serving layer is the end point in the approach of batch computations.
It provides low-latency answers to the clients' queries from the results of batch proccessing of the master dataset.
The serving layer has two main responsibilities: storing of the batch views and creation of indexes for the fast access to those views.
The first issue requires usage of a storage system, that fulfils several specific requirements, as for example fast batch writes and fast random reads. 
Creation of indexes on batch views demands from the serving layer to tackle such problems as for instance denormalization of data.
Each time the batch layer finishes computation of batch views, it stores them into the serving layer, discarding the old ones.
The speed layer recomputes then its indexes.

\mnote{General requirements}
The serving layer has two main metrics of efficiency: throughput and latency.
Throughput implies the number of queries, that can be done in the time unit.
Latency is the time to execute one particular query.
To satisfy both of these properties it is important to design indexes of batch views in a specific way.
For example, in case of storing key-value data, data associated with the key must be collocated on the same machine and stored sequentially.
This makes reading much faster (especially in case of HDD usage), because requires only one disk seek and following sequential read.
Generally, indexes must require access to as few as possible machines for retrieving data, needed for answering query.

\mnote{Denormalization of data}
The serving layer allows to denormalize data of the batch views creating indexes.
This is opposite to the master dataset, where it is possible to keep completely normalized data.
The batch layer does not have to provide low-latency response, and can process such data without any penalty, doing joins or any other data linking.
The serving layer has to return data fast, what imposes a need to make access to indexes as efficient as possible.
Denormalization of data saves from expensive joins.
This makes request time much smaller.

Denormalization leads usually to a redundancy of data.
It also can result in inconsistency of a repeated data pieces, especially when the system becomes complex.
This is, nevertheless, not a problem because of two things.
Firstly, space is not an issue in the serving layer, as long as distributed storage is used, and any number of machines can be added any time.
Secondly, even if there is an inconsistency in data in prepared indexes, it is only temporal, and will be repaired after the next update of batch views.

\mnote{Database requirements}
Database system, that the serving layer uses, must fulfil several requirements.
First of all, it must provide batch writes.
The batch layer recomputes batch views from scratch, and override old ones with newly computed.
It must be possible to write them in a batch fashion efficiently.
The second required property is scalability, that is the database system must be essentially distributed.
It must allow to add as many new machines as needed in case of grow of space requirements.
The third point is ability to make fast random reeds.
This is highly important for a final low-latency query response.
Specifically, fast random reeds of data in prepared indexes gives required efficiency.
The last issue is fault-tolerance, what is a standard property for the Lambda architecture.
It must be provided for the serving layer via distributed database system with replications.

\mnote{Example of a storage system}
An example of a storage system, that can be used for the speed layer is ElephantDB \cite{ElephantDB, Macbeth2013}.
The main properties of ElephantDB is that it is a key-value store, that is simple, easy to use, distributed, and open source.
It fulfils all requirements of the speed layer storage system.
Specifically, it allows to make batch writings, for example as a result of MapReduce computations.
Also, it provides fast read-only random reads.
It does not support random writes, what makes it so simple and almost saves from bugs.