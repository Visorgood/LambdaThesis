\chapter{The Menthal Backend (Speed) Architecture}
\label{chap:menthal_backend_architecture}

\authorsection{Requirements analisys}{SP}

Use cases (quantified self + depression detection) - Use case diagram?
Scope of our work (functional/nonfunctional requirements)
5 pages

Functional requirements

The scope of our work is a part of Lambda Architecture named a Speed Layer.
It requires to implement three main parts: data receiving part, real-time data processing and the store of the results of computations.
The data receiving part do not need to provide any feedback except the acknoledgement that the data was succeessfully received.
Thus on this end of the system we do not need any special API.
On the contrary, the results of computations can be used further by the other components of the Lambda architecture.
Therefore it can be useful to provide an API that allows to request the necessary information.

\mnote{Event receiving}
The data receiving part collects the data that arrives from a message queue.
This data consists of various events that represent the actions a user performs on its smartphone. 
The responsibility of the data receiver is to deserialize incoming messages, identify the types of events and let the processing part to handle them.

\mnote{Aggregations}
The processing part of the Speed Layer performs various aggregations on the incoming data.
The distinctive feature of the Speed Layer is that it works only with limited amount of data, processing only the recent events.
Hence it can calculate aggregations and perform analisys in real time.
The created aggregations have different granularities and characterize the different aspects of user-smartphone interaction.

\mnote{Anomaly detection}
As the events are analyzed in real time, it gives an advantage of detecting various anomalies in incoming data.
We should use an appropriate anomaly detection algorithm that can be applied for the given data.
The system should detect the strange behavior in proper time and immediately notify a user if needed.

\mnote{Results storage}
The results of aggregations are stored while the Batch layer of the Lambda architecture performs its calculations.
During this time they should be available for queries.
Our system should provide an API for these purposes.
When the results of aggreagtions are collected from the Speed Layer and merged with the results from the Batch Layer, they can be deleted from a storage.
 
Non-functional requirements

% 1) interact with other parts (Batch layer, Client app, etc)
% Dependency on other parties
% Documentation
\mnote{Dependency on other parties}
The Speed Layer, as a part of the Lambda Architecture in particular and the Menthal project in general, actively interacts with other components of the system.
As a result, it should meet the following requirements.
First, the Speed Layer needs to be designed taking into account its dependency on other system components.
On the one hand, it should follow the principle of modularity, i.e. be self-contained.
Thus, when other parts of the system change, it should not influence the Speed Layer inner structure.
On the other hand, it should be flexible.
In the case of significant changes in the whole system, the adaptation process to the new environment should be fast and easy.
\mnote{Documentation}
Second, all the available features should be well documented.
As other parts of the system are maintained by other developers, it is necessary to provide them with the most detailed and relevant information.    

% 2) It consists of various technologies
% Extensibility (adding features, and carry-forward of customizations at next major version upgrade)
\mnote{Extensibility}
From the technical side, the Speed Layer consists of a number of cooperating technologies.
These third-party technologies actively develop, the versions of software upgrade.
Therefore our system should allow easy upgrading of its components without a threat of failure.
From the application side, the Menthal project has a great development potential.
New features are constantly added, the functionality is expanded.
Thus our system should meet the extensibility requirement.

% 3) Confidential information
% Security
\mnote{Security}
The essential aspect of our system is its security requirement.
Menthal works with user data and stores personal information of real people.
Therefore it is important to protect the stored information from unauthorized access.
The source of information is a smartphone, that transfers collected data to the server.
The data store on the phone and the communication channel should be protected properly.
However, the security problems of these parts of the system are beyond the scope of our work.
We should provide a secure storage system for the results of aggregations, made on the received user data.
Furthermore, only authorized parties can get an access to these results.

% 3) Real time processing
% Performance / response time
\mnote{Performance}
As the Speed Layer should provide the results of data processing in real time, it puts one more requirement on our system.
It should have good performance characteristics.
In our case the costs of processing cores, physical memory and therefore the number of machines needed to perform calculations is less significant than the system performance.
It should work with the fixed and relatively small response time. 

\mnote{Availability and Reliability}
Availability and reliability are the other important aspects of real-time processing system.
The Speed Layer is the only component that has calculated aggregations on the latest data.
Moreover, for anomaly detection purposes it is important not to miss the significant changes in incoming data and react to them in a timely manner.
Therefore, the system should tend to be always available to provide the results of the performed aggregations.
Also it should constantly process the incoming data without failing to avoid the loss of the significant changes that are needed for anomaly detection. 

\mnote{Fault tolerance}
A real-time system should be fault tolerant.
In the case of Speed Layer it is especially relevant, because it consists of many cooperating components.
Even if several nodes in the cluster crash, it should not break the whole system and interrupt the calculation process.
The Speed Layer should have small recovery time in the case of failure. 

% 5) it uses commodity machines
% Efficiency (resource consumption for given load)
\mnote{Efficiency}
We use commodity machines to construct the cluster where the server part of Menthal runs.
Thus it is necessary for our system to be highly efficient.
It should use the given resources in a proper way, allowing other components of the server part run on the same cluster in the full extend.

% 6) probable increasing load
% Scalability (horizontal, vertical)
\mnote{Scalability}
The system load in the Menthal project depends on two main factors.
First, the increasing number of users auguments the load.
This can happen with the growth of popularity of the Menthal application.
Second, the variety of gathered information about user can be extended.
Hence our system should meet the requirement of scalability.
It should be possible to easily add new nodes in the cluster in the case of increasing load.  

% 7) codability
\mnote{Codability}
During the development we should take into account the codability of the given system.
The amount of the required man-hours to finish the project is an important indicator of a well thought-out system.
In our case there are several different ways to implement the Speed Layer.
Having the same result, some of them require more effort from a programmer than the others.
It is necessary to choose a right strategy to create a working system.

% 8) Other people will maimtain it + it consists of many part
\mnote{Maintainability}
Our goal is to build a working Speed Layer, however later other people will maintain it and perform the further development.
Thus the maintainability of our system is a significant requirement.
It should be easily configurable.
It need to provide a convenient mechanism to perform the tracking of the running tasks, e.g. using log files.
Finally, our system should have necessary tools for testing its functionality.

\section{Design of the system [VI]}

kafka + storm + redis


\section{Data pipeline (from where to where)}
General schema (picture)
3 pages

\section{Algorithms}

\section{Mapping to hardware}
how many servers, characteristics, data transferring
1 page