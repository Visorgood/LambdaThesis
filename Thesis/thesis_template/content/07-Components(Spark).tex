\subsection{Spark [VI]}

Spark is a framework for distributed data processing \cite{Zaharia2010} \cite{Zaharia2013} \cite{Spark1} \cite{Spark2}.
It provides a tool to work with large datasets and streams of data, and to make complex queries on this data.
Spark has several abstractions for representation of data and streams.
The first one - Resilient Distributed Dataset (RDD) - is an object, distributed in the cluster, that contains data to process.
Another one is a Shared Variable - variable, that can be used among the cluster as a counter or lookup table.
One more abstraction is a Discretized Stream (DStream) - representation of a data stream also in a distributed fashion.

\subsubsection{Data model}

To initialize Spark you first create a SparkContext object, that is responsible for a connection of your program to Spark.
It allows to specify properties of your application, and also options of how Spark should run, for example in local mode or on the cluster.
SparkContext gives an access to different parameters and properties of execution environment.

\textit{Resilient Distributed Datasets}\mnote{Resilient Distributed Datasets} (RDDs) is a main abstraction in Spark, that represents dataset in a distributed fashion.
RDD is essentially a readonly collection of elements.
It is fault-tolerant, so that if one partiotion is lost, the whole collection can be recovered.
RDD does not need to exist physically on the cluster nodes.
Instead, it is a lazy object, that can lay in a robust data storage, and be computed on the fly, when computations require particular pieces of data.

There are two methods how to obtain an RDD object: parallelizing exisiting collection in the driver program, and using external dataset.
Existing collection is any collection of data, e.g. array, list, set, etc., that you in fact have in the program.
When you create an RDD object in this way, elements of a collection are copied to a distributed dataset, that you can use than in parallel.
You can specify the number of slices, that is the number of tasks, each machine in the cluster will then execute for this collection.
External dataset is an external file.
It can reside in the local file system, and in the distributed file system like HDFS.
The simple example of what you can do with the external text file, is to count the sum of lines' lengths using functions $map$ and $reduce$ of an RDD object. 
Additionally, you can obtain RDD by transforming another RDD, or by changing persistance of an existing RDD, but this methods are derived in some sense.

RDD supports two types of operations: \textit{transformations} \mnote{transformation} and \textit{actions}\mnote{action}.
Trnsformation creates a new RDD object from existing one.
An example is a $map$ function, that processes each element of an RDD object using specified function, and returns a new RDD object as a result.
Action executes computations on an RDD object, and returns a value.
$Reduce$ is an example of action.
It aggregates all elements giving in the RDD object and return the resulting value.

Transformation in Spark is a lazy operation, in the sense that it is computed only when action operation requires its result to produce output.
This makes execution more efficient when there is a chain of transformations before final action, because your application does not receive then intermediate RDD objects, but only final resulting value, that is usually much smaller.
Nevertheless, there are cases, when you want to compute different actions on the same transformation.
Then it is meaningful to have RDD object of this transformation computed once, and to have a handle to it in your program.
For this case there is a method $persist$, that allows to materialize RDD object.
This is also possible to persist RDD object on disk.

Next we present a simple program, that counts the sum length of all lines in a text file:
\begin{verbatim}
JavaRDD<String> lines = sc.textFile("data.txt");
JavaRDD<Integer> lineLengths = lines.map(s -> s.length());
int totalLength = lineLengths.reduce((a, b) -> a + b);
\end{verbatim}
Example is taken from \cite{Spark1}.
Here we create an RDD object from external file, set $map$ function to count length of the line, and set $reduce$ function to sum up lengths of lines.
Execution starts only when $reduce$ function is called, because, as we discussed, transformations are lazy in Spark.

Normally, when you pass arguments to any function, that executes on the nodes of the cluster, they are simply copied and there is no feedback to the driver program.
Sometimes it is useful to have global variable or lookup table, that all nodes can access.
Spark supports the notion of \textit{shared variable}\mnote{shared variable}.
There are two types of shared variables: \textit{broadcast variables} \mnote{broadcast variable} and \textit{accumulators}\mnote{accumulator}.
Broadcast variable represents readonly value or dataset, that is useful for all nodes as a lookup table or global predefined value.
It is copied to every node using method $broadcast$ of $SparkContext$.
There are efficient algorithms in Spark to make this transfer fast.
Accumulator is distributed counter, that allows all nodes to add up to the global numeric variable.
It can be created using method $accumulator$ of $SparkContext$.
No node can read this value or do anything else than incrementation, what makes its implementation easy and fast.
Only driver program is able to read accumulator's value.

\subsubsection{Streaming model}

StreamingContext
Discretized Streams (DStreams)
Input DStreams
Transformations on DStreams
Output Operations on DStreams
Caching / Persistence, Checkpointing, Deploying Applications, Monitoring Applications ???
Performance Tuning ???
Fault-tolerance Properties ???