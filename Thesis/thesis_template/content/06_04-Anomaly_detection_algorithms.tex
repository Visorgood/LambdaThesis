\section{Anomaly detection algorithms}

% [Anomaly Detection : A Survey]
\textit{Anomaly} is the data that does not correspond to any expected data patterns.
The presence of anomalies in a data flow can be a sign that something has happened with a source of data.
For example, anomalies are caused by a malicious software that changes the system behavior. 
Anomalies in data, collected from medical sensors, can indicate the drastic changes in a patient state of health.
Moreover, an anomaly in sensor data can be a forerunner of a catastrophe, like flood or even a nuclear power plant crash.
Therefore, it is necessary to use an anomaly detection mechanism to prevent undesirable situations.

It is important to distinguish between \textit{anomaly}, \textit{noise} and \textit{novel data}.
Noise is a deviation from normal patterns, that is not interesting for analysis.
Ideally all the noise should be removed before analyzing the information.
Novel data appears when the pattern changes.
This data arrives continuously and leads to creation a new pattern that is later included into the normal model.  

Anomaly detection process is accompanied by the following difficulties.
Usually it is difficult or event not possible to name all the patterns that correspond to the normal behavior.
The boundary between normal and anomalous data can be uncertain.
The behavior that is considered to be normal can change in the course of time.
It is necessary to get rid of noice, that can look like anomalies in some cases.
The degree of deviation from the normal behavior can be different in different situations.
Sometimes anomaly differs from the expected pattern slightly, while in other case the difference can be huge.
Finally, the malicious software often tries to imitate the normal behavior to obstruct its detection. 

Anomalies can be divided into three categories.
The simplest one is a \textit{point anomaly}.
It is an individual instance that differs from all other instances in a given data flow.
Most of the time anomaly detection works with this category of anomalies.
The second one is a \textit{contextual anomaly}.
In this case the instance is analysed taking into account its neighboring instances.
It means that if such data instance occures without a particular context, it is not considered to be an anomaly.
The third category is a \textit{collective anomaly}.
It appears only when the data instances in a data set are related to each other.
For example, a collective anomaly is an appearance of the same data instance in a row, while usually they should be separated by other instances.

Different approaches of anomaly detection exist.
They are grouped into three types: supervised, semi-supervised and unsupervised approaches.
For using a supervised mode, one needs two labeled sets: the normal data and anomalous data.
If a system receives an unlabeled instance, it compares this instance to existing classes and makes a prediction to which class it belongs to.
The problem with supervised approach is that usually the anomalous data set is much smaller than the normal one.
Moreover, as anomaly occures accidentally, it is not possible in some cases to have representative lables for this class of instances.
 
The semi-supervised anomaly detection is almost the same as the supervised, with the exception that it has lables only for the normal instances.
In this case the received data is compared to existing normal pattern and the conclusion is made whether this instance is an anomaly or not.
Theoretically it is possible to use such technique vice versa, having the lables only for anomalous data.
However, often it is hard to predict all the anomalies that can occure in a given system.

The unsupervised approach does not need a predetermined data classes.
This technique is based on the assumption that anomalies occure rarely, comparing to the normal data instances.
Because this approach does not use a training set, it is the most popular anomaly detection mechanism.
This assumption makes possible to adapt a semi-supervised way to an unsupervised one.
For this purpose all the available data os labeled as a normal class.




- detect intrusion(���������)



maybe into practical part of our Thesis:

- we download programs from official/unofficial stores and we click 'agree' to all the access permissions the application ask us to provide
 
- virus signs:
a lot of sms/mms to many or to all of your contacts (to spread a malware) 
sms/calls to specific destinations (when you got extra charged)
fast battery discharging
it making worse device performance 


 


