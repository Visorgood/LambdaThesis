\chapter{Big Data Architecture [SP]}
\label{chap:big_data_architecture}

Traditionally, the process of making decisions is expensive and experiences significant error.
Simple reasoning is a straightforward approach, but it can work only till the certain point.
The main problem is that it can be easily ruined by the wrong assumptions.
Hence one would preferably use evidence based approaches that built on the accumulated data.
The most commonly used tools for this purpose are surveys and experiments, carried out on a control group.   
However, they both have significant disadvantages that makes the process of making decisions more complicated. 
On the one hand, surveys and experiments involve human resources, what leads to high expenses.
On the other hand, these approaches are also vulnerable to errors.
A lot of various reasons can cause an error, such as a survey composed in a wrong way, or a not representative control group.
 
Fortunately, this situation has changed dramatically in recent years.
First of all, data has become available for free, as a by-product of other processes (such as log files).
Moreover, constant reduction of data storage costs allows to warehouse enormous amounts of information, without troubling about the size limits.
Owing to the progress in the information technologies area, both transferring and processing of huge data volumes become easy.
All this gives us an alternative solution to the problem of making decisions that avoids the drawbacks of the strategies presented above.
The name of this solution is Big Data.

The main distinguishing feature of Big Data is that data collection is independent of use case.
Information is collected because it is available and cheap, with the hope that later on it can be used to answer a question that has not arisen yet.
For example, Facebook stores all available data about the users, like demographic information, geographic location, connections with other users, visited websites, clicked links, etc.
As a result it has a huge amount of data, which, with the right approach, can give lots of useful information. 
For instance, afterwards it can be used in targeted advertising, or for performing the social network analysis.

\section{Big Data}
\mnote{Big Data}
There is no consensus about the origins of the term "Big Data", but most of the sources claim that it is first mentioned in the press in 2008.
People start actively using it since 2009 and it spreads quickly owing to its precise and capacious meaning. 
Big Data is characterized by its high (i) variety, (ii) velocity and (iii) volume. [reference - find from wikipedia]
All these three "v" constitute a criterion that allocates Big Data observation into a distinct sector of computer science, which requires ad hoc decisions and a special approach.

\mnote{Variety}
First, Big Data sources are highly diverse.
They differ in the type of produced data - it can be text, images, sounds, raw feed incoming directly from sensors, etc.
Each of these types, in turn, may have a different format.
For instance, text can be transmitted in various languages, coding, formatting and so forth.
Big Data sources also differ in the speed of data flow and the data purity.  
Some of the sources generate noisy information, while others can produce data, that does not need cleaning.
Moreover, Big Data differ in the way how it is collected, how urgently it should be processed and which storage capabilities are available for its warehousing.

\mnote{Velocity}
The diversity of Big Data sources causes the high velocity of data input flow.
For example, Menthal, mentioned above, currently receives data from 50 000 users.
It sums up to almost 30Gb input data per day.
Therefore, Menthal reseives on average around 24Mb per minute, that sends a challenge how to handle input data flow on such a hight speed.

\mnote{Volume}
In the end, massive sources variety, multiplied by high velocity of data generation, results in its enormous size.
For instance, by the end of 2013, the number of Facebook users reaches 1.23 billion.
Each of them not only has some profile information, but also communicates with other users, shares data, updates the timeline and so forth.
In total 2.5 billion content items are shared every day.
Let us assume that each of these events is stored as a JSON object and needs 2Kb on average.
That means that in the end of the year Facebook deals with 4.65Tb * 365 = 1.65Pb of information.
And it is only metadata, not including images and video files that require significantly more space. 
As a result, Facebook deals with storing and processing petabytes of data.

Another example is the information received from sensors.
Sensor is a converter that measures and transforms physical quantity into a digital signal.
Sensors find an application in various fields: manufacturing industry, transportation systems, meteorology, medicine, even modern smartphones have lots of sensors.
The key feature of a sensor is that often it does its job constantly, continuously producing the flow of information, what leads to the large volumes of data.
Nest Labs is an American company that manufactures sensor-driven thermostats and smoke detectors.
The population of United States is about 318 million, so if every hundredth resident uses at least one of Nest thermostats in the house, it sums up to 3.2 million devices.
One thermostat has a variety of sensors, like activity, temperature, humidity, illumination, etc.
Each thermostat generates and transmits around 2Mb of data per day.
Consequently, all Nest thermostats in United States generate around 2.17Pb of data per year. 
The ability to process large amounts of information is the main benefit of Big Data analytics, since with its vast volume it is possible to construct better models.

\mnote{Batch Processing}
There are two fundamentally different ways of processing the Big Data, namely Batch and Real-Time data processing.
In the first case, data is handled in batches, i.e. process collects the data until the batch size is obtained, and only after this the process can perform the necessary actions on a batch as a whole.
This gives several advantages.
Batch processing can be done in the appropriate time, when the computing resources are less busy.
Furthermore, one can set a priority for each task, beginning with more urgent operations.
There is no need in a close supervision of a run, batch processing is mostly autonomous.
Finally, it becomes possible to process multiple operations in one request, instead of handling each operation individually.
That makes data treatment more efficient. 

To give a naive example how batching improves the performance, let us describe the problem of I/O operations.
For instance, an application every second receives data that should be written to the hard disk. 
There are two options how to implement the writing procedure.
On one hand, an application can write the input data every time it receives it, i.e. in our case every second.
On the other hand, it is possible to accumulate input data in memory until it reaches the defined size and than write the whole batch at a time.
It is known that I/O operations involve physical movement of mechanical devices (e.g. seek motion of hard drive).
Thus, the speed of sequential writes to a file is higher than random writes, because in the latter case additional time is spent for seek operations between each write.
This means that in our simple example the second option is preferable, because it decreses the number of seek operation and therefore makes the writing faster.    
The same principle also works in general case, i.e. combining multiple operations in one batch can significantly enhance performance.
However, Batch processing has one significant drawback: the results are always obtained with an arbitrary time delay.

\mnote{Real-Time Processing}
Real-time processing handles data at the moment of arrival. 
The advantage of the latter is that the results are ready almost immediately, which can be an essential requirement in such areas as medicine or security threat prediction. 
In spite of the fact that a certain delay is nevertheless exists, its duration is predetermine and is guaranteed to have a specified value.
That differentiates real-time processing from batch processing.
This fixed delay length varies depending on the application.
In some cases 10 minutes to perform all the computations is still considered to be a real-time processing.
However, in other cases, more than 1 second delay is unacceptable.
Real-time processing makes the information always available and up-to-date.

These significant advantages results in rising popularity of real-time processing, despite the fact that it requires greater effort to design and maintain.
Real-time processing can help to improve traffic in metropolitan areas, offering various travel alternatives for a vehicle, basing on analysis of incoming data about the situation on the roads.
Rapidity of data processing can be necessary in other cases as well.
Sometimes high processing speed can even be indispensable to life, when using in medicine, for example.
Special systems monitor the state of a patient, immediately alerting caregivers in the case of dangerous anomaly occurrence. 

\section{Architectural Requirements}

%System must provide information having gathered data from users and other sources.
%Application of complex algorithms is often necessary.
%Amount of data and rate of its arrival are heavy.
%Those aspects makes system to preprocess indices and aggregations that provide useful information.
%Batch processing is a good solution for that, nevertheless it has a drawback - execution time is long.
%Incremental processing resolves this issue.
%In combination these two approaches allow to design a system, that answers user queries with low latency as well as high accuracy.

%The purpose of the system is to answer queries having data.
%Let's consider an artificial example.
%Suppose we have a website where people pose programming questions, and other answer them.
%In this case the simplest query to the system is to return list of answers for a particular question post.
%Another query is to find all posts containing given keyword.

%One query is easy to answer, another requires application of complex algorithms.
%It is pretty easy to get list of answers to the question post having its id.
%You simply create hash-table that maps questions' ids to lists of answers' ids.
%Search by keyword, or even phrase search, is much more complex.
%To make it possibe we have to build specific inverted index, and this requires much more time to execute and to programm.
%Let's further assume, that our system has to provide keyword search, using precomputed inverted index. 

%Amount of data gathered with the time, as well as intense of arrival, can be huge.
%Let's consider again our example website.
%Assume that on average every second 5 questions and 20 answers appear.
%Every post is about 100 words, each of about 8 unicode symbols.
%The rate of incoming data is then $25*100*8*2=39$ KB per second.
%It is about 3.2 GB per day.
%This leads our system to be able to process such amount of data efficiently, as well as to rapidly reflect the state of inverted index with newly arrived posts.


There are some requirements that are common for most of the Big Data architectures.
Big Data systems deal with a large volumes of information, that constantly growth, therefore a system should be highly scalable.
Scalability has a direct impact on performance, thus it is essential to maintain a system performance on a proper level.
Moreover, availability and reliability are important attributes of a distributed architecture.
All these properties are described in the following paragraphs.

\mnote{Performance}
Performance is a quantitative characteristic of operation speed.  
This concept includes a variety of aspects, such as response time, processing speed, latency, bandwidth, etc.
With regard to extremely high volume and velocity of Big Data the question of system performance is a big issue.
For fair comparison of multiple systems performance a benchmark is used.
A benchmark is a sequence of tests that helps to estimate the performance of the system.

\mnote{Scalability}
Scalability indicates the property to handle an increasing amount of work.
It means that the performance of the system can be enhanced using additional hardware resources.
There are two types of scaling, namely vertical and horizontal.
Vertical scaling means that the single node of a system is enriched, e.g. CPU is added to a computer. 
Horizontal scaling denotes the enlargement of a system by adding new nodes.

\mnote{Commodity hardware}
In the context of Big Data architecture the latter method is of great interest for us.
First, in some cases the system should be distributed geographically.
For instance, adding new nodes closer to the customer can reduce the network load.
Second, because of the decreasing computer price it becomes possible to build highly performant systems using commodity machines.
A commodity computer is a moderately priced machine that is widely available for purchase.
The usage of inexpensive hardware helps considerably decrease the cost of the system.
It is especially relevant in the context of Big Data because of its overwhelming scales.
One of the well-known examples is the Google Lego server.
In 1996 two students, Larry Page and Sergey Brin, needed a cheap but capacious server to test the Pagerank algorithm on a huge data. 
[picture?]
They assembled it using 10 drives 4Gb each and a Lego enclosure. 
Nowadays Google uses commodity computers for building their computing clusters.

\mnote{Fault tolerance}
The direct consequence of the cheap hardware is the high failure rate.
Moreover, the large number of components also increases the probability of failure. 
Thus the system should be highly fault tolerant, with timely error detection and easy automatic recovery.

\mnote{Large files}
The world of Big Data introduces its own specific requirements for architecture design.
The file size can be enormous comparing to the standards.
It is not rare to work with a file of several gigabytes.
Storing the data in large files simplifies data processing.
The size of data itself is huge and it is more efficient to work with several large files than with great number of small files.

There are two metrics to measure the robustbness of a system, namely availability and reliability.
At first sight they look similar, however these two metrics assess a system from different perspective.
\mnote{Availability}
Availability means that a system operates properly at any given moment.
It can be calculated by the following formula: (total time - down time)/total time.
Consequently, availability depends on the sum of the time the system was down.
That means that even if the system fails every hour, but for negligible time, it is still considered to be highly available.

\mnote{Reliability}
In contrast, reliability denotes the capability of a system to operate continuously without failing.
Reliability and availability are the opposite concepts.
The highly available system mentioned above is not reliable, because the intervals of working without failing are relatively short.
However, the system that is down for one hour but only once per month can be reffered to sufficiently reliable.

Big Data architectures have to be targeted to each specific scenario.
For instance, architecture, designed for processing video data from a web camera, differs significantly from one for handling server log files.
Menthal, mentioned in Chapter 2, deals with data, that consists of a bulk of key/value pairs. 
As a result, Big Data technology becomes an umbrella of various systems. 
Data has to be collected, processed, transmitted, stored, protected from attacks, etc.

The Figure~\ref{fig:big_data_flow} shows the general flow of data within the Big Data concept.
Each of the presented steps involves a batch of technologies.
For example, depending on data type and size, one can choose SQL (MySQL, Oracle, Teradata, etc.) or NoSQL (Cassandra, MongoDB, Apache HBase, etc.) solutions for storing Big Data.
Similarly, depending on the application, Real-Time processing (Storm, Spark Streaming, etc.) or Batch processing (Apache Hadoop, etc.) technologies are used. 
Thereby, it is apparent that no one general solution exists for every Big Data problem.
Further we describe the existing Big Data architectures in the context of Menthal needs.

\begin{figure}
  \centering
  \includegraphics [width=0.8\textwidth]{images/big_data_flow}
  \caption{Big Data Flow}
  \label{fig:big_data_flow}
\end{figure}

\section{Naive Approach}
\mnote{PostgreSQL}
It is natural to start with a naive approach, using widely available and easy to use technologies.
As a storage system, one could use a relational database, such as MySQL or PostgreSQL.
PostgreSQL is an object-relational database management system.
It is an open source project, that is quite popular among developers due to its high security standards, good performance and scalability.
PostgreSQL was chosen as a database for Menthal application as a compromise between widespread but primitive MySQL and such large-scale solutions as Oracle or MongoDB.
On the one hand, it supports sufficiently complicated queries and easily scales, comparing to simple MySQL database.
On the other hand, it is not overload with redundant functional and does not require high maintenance affords as large-scale database management systems.

\mnote{REST}
Menthal uses a Representational state transfer (REST) API for communication between client and the database.
In this context RESTful API means that all the required information is transferred as parameters inside of an HTTP request (GET or POST).
The application uses stateless type of data exchange, that means that the server does not store a client state.
It helps to decrease the load on the server side, however increasing the amount of exchange traffic.
This straigforward approach is easy to implement, however, at a certain point, it cannot anymore sustain a constantly growing load.
The datatbase cannot cope anymore with the increased input flow, returning a timeout error.

\mnote{Database mirroring}
Database mirroring avoids the database overload.
It is a technique of keeping redundant copies of a database.
It enhances data availability and allows to always have an accessible copy of a database on one of the machines.
There is a special kind of mirroring called a hot standby, when every change is copied immediately to the mirrored instance.
In this case, there is a guarantee that all the instances keep the identical data, however, it can lead to a high latency.
If some data loss is permitted, a warm standby mode can be used, when the data is not fully synchronized across the database instances.
It does not grant the same information consistency as a hot standby mode, but it shows better performance results.

\mnote{Database caching}
Database caching is another technique for improving architecture scalability. 
In this case data is cached in high-performance store, e.g. in memory.
The access to memory is faster than the reading from a file or a database or transfering data via HTTP.
When the application often uses such external resources, database caching significantly increases performance. 
The drawback of both approaches is that they only improve reading, leaving the problem of large-scale writing unsolved. 

\mnote{Sharding}
In this situation it is reasonable to use horizontal partitioning, also known as sharding.
Sharding helps to enhance write scalability.
It denotes the process of partitioning of database tables by rows.
Each partition (shard) can be settled on separate machine.
This technology allows to spread the write load between multiple database servers.
Also it reduces the index size, thereby improving performance.
However, with growing input flow, the maintaining of shards and auxiliary infrastructure becomes more and more complex, demanding too much effort from developers. 	   
Hence the biggest IT corporations conduct their own research in this area, designing specific architectural solutions for working with Big Data.