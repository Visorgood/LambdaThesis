\subsection{Redis [VI]}

Redis is an open source key-value in-memory data storage \cite{Seguin2012} \cite{Redis}.
It affords blazingly fast and simple tool for maintaining data inside of a single application or a cluster.
Redis is very easy to deploy, learn and use.
It provides 5 data structures, e.g. string, list, set, ordered set and hash, that are useful for different tasks, and give a powerful tool in combination.

\subsubsection{Basics}

Redis is a key-value store, that allows to hold all data of your application.
Redis lets to create many databases and switch between them.
For one database it maintains one global map of keys to values.
Although Redis is in-memory database system, it swaps data continuously to disk to provide persistance in case of application's failure.
This is also important in a distributed environment to have global data available between applications.

Record in Redis is a key-value pair.
The key is always a string.
It does not contain data, that your application manipulates, but it identifies piece of data, that you want to store.
The value can be of different types, but in any case it is a meaningful peace of data, that you store in the database.

Redis provides many useful and at the same time simple commands to work with data.
The two simplest commands are $SET$ and $GET$, that let to store a key-value pair into database, and to get value by key, respectively.
This is an example of how you can use these commands:
\begin{verbatim}
> SET server:name "SERVER1"
OK
> GET server:name
"SERVER1"
\end{verbatim}
There many commands to work with data structures, that Redis provides.

Redis allows to query only values by keys.
You cannot find a key, this is a crucial difference with classical relational databases.
And if you do not know a key, you can not find a value.
There is a command $keys$, that returns all keys, stored in the database, but it is strongly advised not to use it in production application, because it does a linear scan through all the keys, what can be very slow.
Another point is that operation of receiving a value by a key works in constant time, basically instantly.
This makes Redis blayingly fast and useful, because grow of the database does not affect its performance.

Redis stores the database to disk every minute if at least 1000 keys has been changed.
It makes less swaps, if you change less number of keys.
It does swap completely as a snapshot.
There is an alternative way to set Redis to make appending swaps.

\subsubsection{Data structures}

The basic data structure in Redis is a \textit{String}.
Keys are always strings.
Values can be of any type, but String is the most popular, because it represents atomic piece of data.
As a String you can store not just something simple like name of a user or his password, but also complex objects like JSON object, for example:
\begin{verbatim}
> SET users:user001 '{"firstname": "john", "lastname": "smith"}'
\end{verbatim}
Redis provides standard commands to work with strings, e.g. $STRLEN$, $GETRANGE$, $APPEND$, etc.
If you store numeric value as a string, you can work with it as with integer.
Redis has several useful commands for this case, e.g. $INCR$, $INCRBY$, $DECR$, $DECRBY$, $SETBIT$, $GETBIT$.

The first complex data structure is a \textit{List}.
List is simply an array of values identified by a key.
There are specific commands to work with lists, e.g. $LPUSH$, $RPUSH$, $LRANGE$, $LLEN$, $LPOP$, $RPOP$, etc.
The values of the list can by anything, not only strings.
This gives powerful tool to store complex combined data.

Set

Sorted set

Hash

\subsubsection{Features}

Transactions

Expiration

Publication and Subscriptions

Monitor and Slow Log

Sorting

Scanning

Scripts

\subsubsection{Administration}

