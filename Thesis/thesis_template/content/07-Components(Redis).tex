\subsection{Redis [VI]}
\label{subs:redis}

Redis is an in-memory storage that keeps data in a key-value fashion \cite{Seguin2012, Redis}.
It maintains data handled by a single application or by a cluster.
Redis is easy to deploy, learn and use.
It provides 5 data types, e.g. string, list, set, ordered set and hash that are useful for different tasks, and give a powerful tool in combination.
Redis is an open source database system.

%\subsubsection{Basics}

Redis allows to create many different databases and switch between them.
Within one database Redis maintains a global map of keys to values.
The key is always a string.
It identifies piece of data, and does not represent data itself.

Redis provides many useful and simple commands to work with data.
The two simplest commands are \lstinline{SET} and \lstinline{GET}, that let to store a key-value pair into database, and to get value by key, respectively.
Listing~\ref{listing:RedisSetGet} shows an example of how to use these commands.
In this example the key is \lstinline{server:name} and the value is \lstinline{"SERVER1"}.

\begin{lstlisting}[float=h, caption=Example of usage of commands SET and GET., label=listing:RedisSetGet]
> SET server:name "SERVER1"
OK
> GET server:name
"SERVER1"
\end{lstlisting}

Redis allows to query values only by keys.
This is opposite to relational databases, where it is possible to search everything stored.
There is one exception, a command \lstinline{KEYS}, that returns all keys, stored in the database.
But it is strongly advised not to use it in production release, because it does a linear scan through all the keys, what can be very slow.
This command is rather for administrative issues.
Another point is that operation of receiving a value by a key works in a constant time, basically instantly.
This makes Redis in fact fast and useful, because grow of the database does not affect its performance.

Although Redis is an in-memory database system, it swaps data continuously to disk to provide persistence in case of application's failure.
Redis stores the database to disk every minute if at least 1000 keys have been changed.
It makes fewer swaps, if fewer keys have been changed.
It does swap completely as a snapshot.
There is though an alternative way to set Redis to make appending swaps.

%!!!!!!!!!!!!!!!!!!!!!!!!!!!!!
%Replications (can be written)
%!!!!!!!!!!!!!!!!!!!!!!!!!!!!!

%\subsubsection{Data types}

The basic data type in Redis is \textit{String}.
Keys are always strings.
Values can be of any type, but String is the most popular, because it represents atomic piece of data.
As a String it is possible to store not just something simple like name of a user or text data, but also complex objects like JSON object.
On the Listing~\ref{listing:RedisSETComplexValue} is an example of such action.

\begin{lstlisting}[float=h, caption=Usage of SET command with complex value., label=listing:RedisSETComplexValue]
> SET users:user001 '{"firstname":"john", "lastname":"smith"}'
OK
> GET users:user001
"{\"firstname\":\"john\", \"lastname\":\"smith\"}"
\end{lstlisting}

Redis provides standard commands to work with strings, e.g. \lstinline{STRLEN}, \lstinline{GETRANGE}, \lstinline{APPEND}, etc.
Storing a numeric value as a string, it is possible to work with it as with integer.
Redis has several useful commands for this case, e.g. \lstinline{INCR}, \lstinline{INCRBY}, \lstinline{DECR}, \lstinline{DECRBY}, \lstinline{SETBIT}, \lstinline{GETBIT}.
On the Listing~\ref{listing:RedisIntegerValue} is an example of working with value as with integer.

\begin{lstlisting}[float=h, caption=Working with value as with integer., label=listing:RedisIntegerValue]
> SET x 1
OK
> INCR x
(integer) 2
> INCRBY x 5
(integer) 7
> DECR x
(integer) 6
> DECRBY x 3
(integer) 3
> GET x
"3"
\end{lstlisting}

The first complex data type in Redis is \textit{List}.
List is simply an array of values identified by a key.
There are specific commands to work with lists, e.g. \lstinline{LPUSH}, \lstinline{RPUSH}, \lstinline{LRANGE}, \lstinline{LLEN}, \lstinline{LPOP}, \lstinline{RPOP}, etc.
Values of the list can be anything, not only strings.
This gives powerful tool to store complex combined data.
Listing~\ref{listing:RedisListCommands} shows basic usage of Lists.

\begin{lstlisting}[float=h, caption=Usage of List data type commands., label=listing:RedisListCommands]
> LPUSH users john mike jack
(integer) 3
> LRANGE users 0 -1
1) "jack"
2) "mike"
3) "john"
\end{lstlisting}

The next data type is \textit{Set}.
Set is an unordered array of distinct values.
There are many useful operations to work with sets in Redis, e.g. \lstinline{SADD}, \lstinline{SISMEMBER}, \lstinline{SINTER}, \lstinline{SINTERSTORE}, etc.
Set is implemented internally as a hash-table, hence, it provides constant lookup.
Listing~\ref{listing:RedisSetCommands} shows basic usage of Sets.

\begin{lstlisting}[float=h, caption=Usage of Set data type commands., label=listing:RedisSetCommands]
> SADD friends:john mike paul jack tania
(integer) 4
> SISMEMBER friends:john paul
(integer) 1 
\end{lstlisting}

\textit{Sorted set} is an extension of a regular Set, that provides order for its elements.
To order values, it uses scores.
Each value has then a score, that must be set together with inserted value.
It is possible then to lookup all values in the specific range of the score, or obtain all values in the sorted order.
Main commands for working with Sorted sets are \lstinline{ZADD}, \lstinline{ZCOUNT}, \lstinline{ZRANK}, \lstinline{ZREVRANK}, etc.
In case when it is needed to get array sorted by integers, Sorted set is a perfect data structure to do that.
Listing~\ref{listing:RedisSortedSetCommands} shows basic usage of Sorted sets.

\begin{lstlisting}[float=h, caption=Usage of Sorted set data type commands., label=listing:RedisSortedSetCommands]
> ZADD friends:john 70 mike 95 paul 92 jack 75 tania 1 dave
(integer) 5
> ZCOUNT friends:john 90 100
(integer) 2
\end{lstlisting}

\textit{Hash} is the last data type given in Redis.
Hash is essentially a hash-table, that maps strings to any objects.
It works also in constant time.
There are different commands to work with Hashes, e.g. \lstinline{HSET}, \lstinline{GHET}, \lstinline{HMSET}, \lstinline{HMGET}, \lstinline{HGETALL}, \lstinline{HKEYS}, \lstinline{HDEL}, etc.
Hashes give simple useful tool to store objects in an object-oriented way.
For example, instead of storing information about user as a JSON object, or as another string representation, it can be done using Hash.
Every field of a user is then put as a separate key-value pair.
Listing~\ref{listing:RedisHashCommands} shows basic usage of Hashes.

\begin{lstlisting}[float=h, caption=Usage of Hash data type commands., label=listing:RedisHashCommands]
> HSET users:john id 375
(integer) 1
> HSET users:john name "john"
(integer) 1
> HSET users:john lastname "smith"
(integer) 1
> HGETALL users:john
1) "id"
2) "375"
3) "name"
4) "john"
5) "lastname"
6) "smith"
\end{lstlisting}

%\subsubsection{Features}

Redis allows to use transactions to make several operations atomic.
Basically, any single operation that Redis does, is atomic, because Redis is single-threaded server.
On execution of any single command, no other command can start execution before the first one is in progress.
But sometimes it is useful or even necessary to do several operations in line as one atomic command.
For that case Redis provides transactions.
Listing~\ref{listing:RedisMultiExec} shows usage of transactions.
All commands between \lstinline{MULTI} and \lstinline{EXEC} is one the atomic operation.

\begin{lstlisting}[float=h, caption=Usage of commands MULTI and EXEC., label=listing:RedisMultiExec]
MULTI
...
EXEC
\end{lstlisting}

Redis allows to set expiration time for a key, or set an expiration point in time using Unix timestamp.
This can be done using commands \lstinline{EXPIRE} and \lstinline{EXPIREAT}.
It does not important, what data structure a key identifies.
When a key expires, it does not exist in Redis anymore.
To check expiration time for a key there is a command \lstinline{TTL}.
To remove expiration from a key, command \lstinline{PERSIST} exists.
Listing~\ref{listing:RedisExpire} shows usage of commands for setting expiration time.

\begin{lstlisting}[float=h, caption=Usage of commands EXPIRE and EXPIREAT., label=listing:RedisExpire]
> SET user001 "john smith"
OK
> EXPIRE user001 60
(integer) 1
> EXPIREAT user001 14572436115
(integer) 1
\end{lstlisting}

There is a mechanism of subscription and publication in Redis.
To subscribe to a channel, command \lstinline{SUBSCRIBE} exists.
When somebody publishes something to this channel using command \lstinline{PUBLISH}, subscribers receive this message.
This tool allows to notify different nodes in the system, that connect to Redis database, about events.

Redis has a useful command \lstinline{SORT}, that allows to sort lists, sets and sorted sets by their values.
Applied to a collection, this command returns elements in a sorted order.
There are different parameters to inform this command how to execute sorting.
Listing~\ref{listing:RedisSort} shows a simple example of how to use command \lstinline{SORT}.

\begin{lstlisting}[float=h, caption=Usage of SORT command., label=listing:RedisSort]
> RPUSH simple_array 3 12 7 1 4 8 2 9
(integer) 8
> SORT simple_array
1) "1"
2) "2"
3) "3"
4) "4"
5) "7"
6) "8"
7) "9"
8) "12"
\end{lstlisting}

Redis allows to use Lua scripts.
It is possible to write a script, to store it, and then to use directly as any Redis command.
This feature gives an opportunity to write own commands, that contain several actions, logically united in one meaningful operation.
Such defined procedure executes in atomic way, what is another advantage.