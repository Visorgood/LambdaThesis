\usepackage[bottom=4cm, right=4cm]{geometry}

%~~~~~~~~~~~~~~~~~~~~~~~~~~~~~~
%% LOAD FIRST
%~~~~~~~~~~~~~~~~~~~~~~~~~~~~~~

%% Encoding
\usepackage[latin1]{inputenc}

%% Sprache
%  ftp://tug.ctan.org/pub/tex-archive/macros/latex/required/babel/babel.pdf
\usepackage[english]{babel}
\usepackage{blindtext}
%% Besserer Flattersatz (Linksbuendig, statt Blocksatz)
%  ftp://tug.ctan.org/pub/tex-archive/macros/latex/contrib/ms/ragged2e.pdf
\usepackage{ragged2e}

%% Bilder
%  ftp://tug.ctan.org/pub/tex-archive/macros/latex/required/graphics/grfguide.pdf
\usepackage[]{graphicx}

%% Farben
%  ftp://tug.ctan.org/pub/tex-archive/macros/latex/contrib/xcolor/xcolor.pdf
% Incompatible: Do not load when using pstricks !
\usepackage[table,usenames,dvipsnames]{xcolor} % Load for using rowcolors command in tables


%\usepackage[nointegrals]{wasysym}

%% Mathematik Basispaket
%  ftp://tug.ctan.org/pub/tex-archive/macros/latex/required/amslatex/math/amsldoc.pdf
\usepackage[]{amsmath}



% damit es nicht zu "no room for new \dimen"-Fehler kommt
\usepackage{etex}

\usepackage{tikz}
\usepackage{tkz-graph}
%  \usepackage{ifthen}
%  \usepackage{xstring}
%  \usepackage{calc}
%  \usepackage{pgfopts}

% See http://pgfplots.sourceforge.net/pgfplots.pdf
\usepackage{pgfplots}
\usepackage{pgfplotstable}
\usetikzlibrary{pgfplots.groupplots}
\usetikzlibrary{patterns}

% argument #1: any options
\newenvironment{customlegend}[1][]{%
    \begingroup
    % inits/clears the lists (which might be populated from previous
    % axes):
    \csname pgfplots@init@cleared@structures\endcsname
    \pgfplotsset{#1}%
}{%
    % draws the legend:
    \csname pgfplots@createlegend\endcsname
    \endgroup
}%

% makes \addlegendimage available (typically only available within an
% axis environment):
\def\addlegendimage{\csname pgfplots@addlegendimage\endcsname}

\usepackage{epstopdf}


%prevent number and text in list of figures to overlap
\makeatletter
\renewcommand*\l@figure{\@dottedtocline{1}{1.5em}{3em}}% 3em instead of 2.3em
\let\l@table\l@figure
\makeatother


%~~~~~~~~~~~~~~~~~~~~~~~~~~~~~~
%% TEXT
%~~~~~~~~~~~~~~~~~~~~~~~~~~~~~~

%% Schriftart

%% http://www.tug.dk/FontCatalogue/mathfonts.html



\usepackage{lmodern}
%\usepackage{kpfonts}
%\usepackage{mathptmx}
%\usepackage[sc]{mathpazo}
%\usepackage{charter}

\usepackage[T1]{fontenc}
%\usepackage{ae,aecompl}

%Optischer Randausgleich mit pdfTeX
\usepackage[%
	expansion=false, % if true: better typography, but larger PDF file (and not compatible with all fonts)
	protrusion=true
]{microtype}

%% Zum Unterstreichen
%  ftp://tug.ctan.org/pub/tex-archive/macros/latex/contrib/misc/ulem.sty
\usepackage[normalem]{ulem}

%%% Doc: ftp://tug.ctan.org/pub/tex-archive/macros/latex/contrib/soul/soul.pdf
\usepackage{soul}	% Unterstreichen, Sperren

%% Control the look & feel of the captions from floating environments like figure and table.
%  ftp://tug.ctan.org/pub/tex-archive/macros/latex/contrib/caption/caption.pdf
\usepackage{caption}
% Aussehen der Captions
\captionsetup{
   margin = 10pt,
   font = {small,rm},
   labelfont = {small,bf},
   format = plain, % oder 'hang'
   indention = 0em,  % Einruecken der Beschriftung
   labelsep = colon, %period, space, quad, newline
   justification = RaggedRight, % justified, centering
   singlelinecheck = true, % false (true=bei einer Zeile immer zentrieren)
   position = bottom %top
}
%%% Bugfix Workaround (Matthias Pospiech!?)
\DeclareCaptionOption{parskip}[]{}
\DeclareCaptionOption{parindent}[]{}

%% �berschriften komlett Umdefinieren
%  ftp://tug.ctan.org/pub/tex-archive/macros/latex/contrib/titlesec/titlesec.pdf
\usepackage{titlesec}

%% Mehrere Text-Spalten
\usepackage{multicol}

%%
% Doc: ftp://tug.ctan.org/pub/tex-archive/macros/latex/contrib/paralist/paralist.pdf
\usepackage{paralist}

%% Better than 'paralist' and 'enumerate' because it uses a keyvalue interface !
%  Do not load together with package 'enumerate'.
%  ftp://tug.ctan.org/pub/tex-archive/macros/latex/contrib/enumitem/enumitem.pdf
\usepackage{enumitem}

%% Schriftanpassung nach scrguide.pdf
\setkomafont{subsection}{\sffamily}

%% Chapter-Style definieren
% --> Rechtsb�ndig: Gro�e Kapitelnummer, darunter der Name
\titleformat{\chapter}[display]
{\filleft\usekomafont{chapter}\Huge}
{\fontsize{100pt}{50pt}\selectfont\thechapter}
{-2ex} %is vertical space in [display] mode
%Platz vor dem ganzen Krempel
{\vspace{1ex}} %1ex ist die H�he von x im aktuellen Font
%Platz danach
[\vspace{1ex}]


\usepackage{nowidow}


%~~~~~~~~~~~~~~~~~~~~~~~~~~~~~~
%% TABELLEN (tabularx wird von ltxtable geladen!?)
%~~~~~~~~~~~~~~~~~~~~~~~~~~~~~~

%% Tabellen ueber mehere Seiten
%  ftp://tug.ctan.org/pub/tex-archive/macros/latex/contrib/carlisle/ltxtable.pdf
\usepackage{ltxtable} % Longtable + tabularx
                      % (multi-page tables) + (auto-sized columns in a fixed width table)
%% bessere Abstaende innerhalb der Tabelle (Layout))
%  ftp://tug.ctan.org/pub/tex-archive/macros/latex/contrib/booktabs/booktabs.pdf
\usepackage{booktabs}

%% Erweiterte Funktionen innerhalb von Tabellen
%  ftp://tug.ctan.org/pub/tex-archive/macros/latex/contrib/multirow/multirow.sty
\usepackage{multirow} % Mehrfachspalten

%% Tabellen: Ausrichtung an Komma oder Punkt
\usepackage{dcolumn}

\usepackage{pbox}
\usepackage{footnote}

%~~~~~~~~~~~~~~~~~~~~~~~~~~~~~~
%% FARBEN
%~~~~~~~~~~~~~~~~~~~~~~~~~~~~~~

\definecolor{sectioncolor}{RGB}{0, 0, 0}    % Schwarz

% Farbe des Textes
\definecolor{textcolor}{RGB}{0, 0, 0}        % Schwarz

% Farbe fuer grau hinterlegte Boxen (fuer Paket framed.sty)
\definecolor{shadecolor}{gray}{0.90}

% Farben fuer die Links im PDF
\definecolor{pdfurlcolor}{rgb}{0.6,0,0}
\definecolor{pdffilecolor}{rgb}{0,0.4,0}
\definecolor{pdflinkcolor}{rgb}{0,0,0.75}

% Java Syntax Highlighting
\definecolor{javared}{rgb}{0.6,0,0} % for strings
\definecolor{javagreen}{rgb}{0.25,0.5,0.35} % comments
\definecolor{javapurple}{rgb}{0.5,0,0.35} % keywords
\definecolor{javadocblue}{rgb}{0.25,0.35,0.75} % javadoc

% more
\definecolor{darkblue}{rgb}{0.0,0.0,0.6}
\definecolor{cyan}{rgb}{0.0,0.6,0.6}

%colors for teh dbs
\colorlet{EventStoreClr}{SpringGreen}
\colorlet{HBaseClr}{Dandelion}
\colorlet{PostgresClr}{Cyan}
\colorlet{RedisClr}{Apricot}


\definecolor{plot1}{RGB}{128,0,128}
\definecolor{plot2}{RGB}{128,128,128}
\definecolor{plot3}{RGB}{236,217,198}
\definecolor{plot4}{RGB}{255,178,178}
\definecolor{plot5}{RGB}{178,178,255}



%~~~~~~~~~~~~~~~~~~~~~~~~~~~~~~
%% LITERATUR
%~~~~~~~~~~~~~~~~~~~~~~~~~~~~~~

% ftp://tug.ctan.org/pub/tex-archive/macros/latex/contrib/natbib/natbib.pdf
%\usepackage[%
	%round,	%(default) for round parentheses;
%	square,	% for square brackets;
	%curly,	% for curly braces;
	%angle,	% for angle brackets;
	%colon,	% (default) to separate multiple citations with colons;
%	comma,	% to use commas as separaters;
	%authoryear,% (default) for author-year citations;
%	numbers,	% for numerical citations;
	%super,	% for superscripted numerical citations, as in Nature;
%	sort,		% orders multiple citations into the sequence in which they appear in the list of references;
%	sort&compress,    % as sort but in addition multiple numerical citations
                   % are compressed if possible (as 3-6, 15);
	%longnamesfirst,  % makes the first citation of any reference the equivalent of
                   % the starred variant (full author list) and subsequent citations
                   %normal (abbreviated list);
	%sectionbib,      % redefines \thebibliography to issue \section* instead of \chapter*;
                   % valid only for classes with a \chapter command;
                   % to be used with the chapterbib package;
	%nonamebreak,     % keeps all the authors names in a citation on one line;
                   %causes overfull hboxes but helps with some hyperref problems.
%]{natbib}

%% Bibliography styles created with custombib
%  ftp://tug.ctan.org/pub/tex-archive/macros/latex/contrib/custom-bib/makebst.pdf
%\bibliographystyle{bib/bst/AlphaDINFirstName}
\bibliographystyle{alpha}


%~~~~~~~~~~~~~~~~~~~~~~~~~~~~~~
%% BILDER & GRAFIKEN
%~~~~~~~~~~~~~~~~~~~~~~~~~~~~~~

% Stellt die Option [H] fuer Floats zur Verf�gung
\usepackage{float}

% Floats immer erst nach der Referenz setzen
\usepackage{flafter}
 
%% Platz ober- und unterhalb des Bildes
\setlength{\intextsep}{0.5\baselineskip} 

%%% Doc: http://www.ctan.org/tex-archive/macros/latex/contrib/sidecap/sidecap.pdf
\usepackage[%
%	outercaption,%	(default) caption is placed always on the outside side
%	innercaption,% caption placed on the inner side
%	leftcaption,%  caption placed on the left side
	rightcaption,% caption placed on the right side
%	wide,%			caption of float my extend into the margin if necessary
%	margincaption,% caption set into margin
	ragged,% caption is set ragged
]{sidecap}

\renewcommand\sidecaptionsep{2em}
%\renewcommand\sidecaptionrelwidth{20}
\sidecaptionvpos{table}{c}
\sidecaptionvpos{figure}{c}

%% Subfigures
%  ftp://tug.ctan.org/pub/tex-archive/macros/latex/contrib/subfig/subfig.pdf
%  Incompatible: loads package capt-of. Loading of 'capt-of' afterwards will fail therefor
\usepackage{subfig}

% Aussehen der Captions fuer subfigures (subfig-Paket)
\captionsetup[subfloat]{%
   margin = 10pt,
   font = {small,rm},
   labelfont = {small,bf},
   format = plain, % oder 'hang'
   indention = 0em,  % Einruecken der Beschriftung
   labelsep = space, %period, space, quad, newline
   justification = RaggedRight, % justified, centering
   singlelinecheck = true, % false (true=bei einer Zeile immer zentrieren)
   position = bottom, %top
   labelformat = parens % simple, empty % Wie die Bezeichnung gesetzt wird
}

%% Bilder von Text Umfliessen lassen
%  ftp://tug.ctan.org/pub/tex-archive/macros/latex/contrib/wrapfig/wrapfig.sty
% defines wrapfigure and wrapfloat
\usepackage{wrapfig}

% Platz ober- und unterhalb des Bildes
\setlength{\intextsep}{0.5\baselineskip} 
%\setlength{\wrapoverhang}{\marginparwidth} % Overlap des Bildes ...
%\addtolength{\wrapoverhang}{\marginparsep} % ... in den margin
%\setlength{\columnsep}{1em} % Abstand zum Text

% Make float placement easier ???
\renewcommand{\floatpagefraction}{.75} % vorher: .5
\renewcommand{\textfraction}{.1}       % vorher: .2
\renewcommand{\topfraction}{.8}        % vorher: .7
\renewcommand{\bottomfraction}{.5}     % vorher: .3
\setcounter{topnumber}{3}              % vorher: 2
\setcounter{bottomnumber}{2}           % vorher: 1
\setcounter{totalnumber}{5}            % vorher: 3

%%% Doc: http://www.ctan.org/tex-archive/macros/latex/contrib/sidecap/sidecap.pdf
\usepackage[%
%	outercaption,%	(default) caption is placed always on the outside side
%	innercaption,% caption placed on the inner side
%	leftcaption,%  caption placed on the left side
	rightcaption,% caption placed on the right side
%	wide,%			caption of float my extend into the margin if necessary
%	margincaption,% caption set into margin
	ragged,% caption is set ragged
]{sidecap}

\renewcommand\sidecaptionsep{2em}
%\renewcommand\sidecaptionrelwidth{20}
\sidecaptionvpos{table}{c}
\sidecaptionvpos{figure}{c}


%~~~~~~~~~~~~~~~~~~~~~~~~~~~~~~
%% KOPF- UND FUSSZEILEN
%~~~~~~~~~~~~~~~~~~~~~~~~~~~~~~

% ftp://tug.ctan.org/pub/tex-archive/macros/latex/contrib/koma-script/scrguide.pdf
\usepackage[%
   automark,         % automatische Aktualisierung der Kolumnentitel
   nouppercase,      % Grossbuchstaben verhindern
   %markuppercase    % Grossbuchstaben erzwingen
   %markusedcase     % vordefinierten Stil beibehalten
   %komastyle,       % Stil von Koma Script
   %standardstyle,   % Stil der Standardklassen
]{scrpage2}

\renewcommand*{\chaptermark}[1]{%
  \markboth{\chaptermarkformat #1}{}}
\renewcommand*{\sectionmark}[1]{%
  \markright{\sectionmarkformat #1}}


%~~~~~~~~~~~~~~~~~~~~~~~~~~~~~~
%% MATHE & FORMELN
%~~~~~~~~~~~~~~~~~~~~~~~~~~~~~~

%% Bracket Schreibweise
%  ftp://tug.ctan.org/pub/tex-archive/macros/latex/contrib/misc/braket.sty
\usepackage{braket}

%% Durchstreichen
%  ftp://tug.ctan.org/pub/tex-archive/macros/latex/contrib/misc/cancel.sty
\usepackage{cancel}

%% Hervorheben
%  ftp://tug.ctan.org/pub/tex-archive/macros/latex/contrib/mh/doc/empheq.pdf
\usepackage{empheq}


%~~~~~~~~~~~~~~~~~~~~~~~~~~~~~~
%% LISTINGS & CODE & DIAGRAMME (UML)
%~~~~~~~~~~~~~~~~~~~~~~~~~~~~~~

%% Listings Paket
%  http://www.pvv.ntnu.no/~berland/latex/docs/listings.pdf
\usepackage{listings}
\lstloadlanguages{Java,SQL,XML}

%% f�r alle Listings
\lstset{ 
		 backgroundcolor=\color{gray!12},
         basicstyle=\footnotesize\ttfamily, % Standardschrift    
         numbers=left,               % Ort der Zeilennummern
         numberstyle=\tiny,          % Stil der Zeilennummern
%         stepnumber=1,               % Abstand zwischen den Zeilennummern
         numbersep=5pt,              % Abstand der Nummern zum Text
         tabsize=2,                  % Groesse von Tabs
         extendedchars=true,         %
         breaklines=true,            % Zeilen werden Umgebrochen    
         showspaces=false,           % Leerzeichen anzeigen ?%         showtabs=false,             % Tabs anzeigen ?
         showstringspaces=false,     % Leerzeichen in Strings anzeigen ?
		 %frame=single,
		 %frame={l,r,b},
		 rulecolor=\color{black!60},
		 %captionpos=b,
		 xleftmargin=15pt,
		 framexleftmargin=14pt,
		 %framexrightmargin=4pt,
         %framexbottommargin=4pt,
         %framextopmargin=4pt,
         belowcaptionskip=5pt,
 }
 
\DeclareCaptionFont{white}{\color{white}}
\DeclareCaptionFormat{listing}{\colorbox{black!60}{\parbox{0.985\textwidth}{\hspace{13pt}#1#2#3}}}
\captionsetup[lstlisting]{format=listing,labelfont={white},textfont={sf,white},
singlelinecheck=false, margin=1pt, font={bf,small}}

\lstdefinestyle{Java}
{
  keywordstyle=\color{javapurple}\bfseries,
  stringstyle=\color{javared}, % Farbe der String
  commentstyle=\color{javagreen}, % Farbe der Kommentare
  morecomment=[s][\color{javadocblue}]{/**}{*/},
  morekeywords={public,lang}
}

\lstdefinelanguage{XML}
{
  morestring=[b][\color{javadocblue}]",
  morestring=[s][\color{black}]{>}{<},
  %morecomment=[s]{<?}{?>},
  %stringstyle=\color{javadocblue},
  identifierstyle=\color{javagreen},
  keywordstyle=\color{javapurple},
  morekeywords={category,lang}% list your attributes here
}

%% tikz-uml zur Darstellung von UML-Diagrammen
%  http://www.ensta-paristech.fr/~kielbasi/tikzuml/index.php?lang=en
\usepackage{packages/tikz-uml}	% aus dem Ordner packages, da nicht im MikTex-Repository!

%~~~~~~~~~~~~~~~~~~~~~~~~~~~~~~
%% SONSTIGES
%~~~~~~~~~~~~~~~~~~~~~~~~~~~~~~

\setcounter{lofdepth}{1}  % Tiefe Abbildungsverzeichnis, 1 = nur figures, 2 = figures + subfigures
\setcounter{secnumdepth}{2}    % Tiefe der Nummerierung
\setcounter{tocdepth}{2}	   % Tiefe Inhaltsverzeichnis

%% Intelligente Querverweise
%  http://www.ctex.org/documents/packages/bibref/varioref.pdf
\usepackage[english]{varioref}

%% Fussnoten/Endnoten
%  ftp://tug.ctan.org/pub/tex-archive/macros/latex/contrib/footmisc/footmisc.pdf
\usepackage[
   bottom,      % Footnotes appear always on bottom. This is necessary
                % especially when floats are used
   stable,      % Make footnotes stable in section titles
   perpage,     % Reset on each page
   %para,       % Place footnotes side by side of in one paragraph.
   %side,       % Place footnotes in the margin
   ragged,      % Use RaggedRight
   %norule,     % suppress rule above footnotes
   multiple,    % rearrange multiple footnotes intelligent in the text.
   %symbol,     % use symbols instead of numbers
]{footmisc}

%% Advanced features for clever quotations
%  ftp://tug.ctan.org/pub/tex-archive/macros/latex/contrib/csquotes/csquotes.pdf
\usepackage[%
   babel,            % the style of all quotation marks will be adapted
                     % to the document language as chosen by 'babel'
   german=quotes,		% Styles of quotes in each language
   english=british,
   french=guillemets
]{csquotes}

\renewcommand*{\mkblockquote}[4]{\enquote{#1}#2#4#3}


% Use �\cite{NEEDED}� to get Wikipedia-style �citation needed� in document
\usepackage{ifthen}
\let\oldcite=\cite
\renewcommand\cite[1]{\ifthenelse{\equal{#1}{NEEDED}}{\ensuremath{^\texttt{\color{red}[citation~needed]}}}{\oldcite{#1}}}

%\usepackage[disable]{todonotes}
\usepackage{todonotes}

% margin notes aligned with paragraph
\newcommand{\mnote}[1]{{\hspace{0pt}\marginpar[\raggedleft\emph{\small{#1}}]{\raggedright\emph{\small{#1}}}}\ignorespaces}


%~~~~~~~~~~~~~~~~~~~~~~~~~~~~~~
%% LINKS (hyperref am Besten zum Schluss Laden laut Doku)
%~~~~~~~~~~~~~~~~~~~~~~~~~~~~~~

%% Setzen von URLs. In Verbindung mit hyperref sind diese auch aktive Links.
%  ftp://tug.ctan.org/pub/tex-archive/macros/latex/contrib/misc/url.sty
\usepackage{url}
\urlstyle{sf}

%% Basispaket hyperref
%  ftp://tug.ctan.org/pub/tex-archive/macros/latex/contrib/hyperref/doc/manual.pdf
\usepackage[
   % Farben fuer die Links
   colorlinks=false,%true,         % Links erhalten Farben statt Kaeten
   urlcolor=pdfurlcolor,    % \href{...}{...} external (URL)
   filecolor=pdffilecolor,  % \href{...} local file
   linkcolor=pdflinkcolor,  %\ref{...} and \pageref{...}
   citecolor=pdffilecolor,	% Farbe von Links zum Literaturverzeichnis
   % Links
   raiselinks=true,			% calculate real height of the link
   breaklinks,              % Links �berstehen Zeilenumbruch
   backref=page,            % Backlinks im Literaturverzeichnis (section, slide, page, none)
   pagebackref=true,        % Backlinks im Literaturverzeichnis mit Seitenangabe
   verbose,					% extra diagnostic messages are printed in the log file
   hyperindex=true,         % backlinkex index
   linktocpage=true,        % Inhaltsverzeichnis verlinkt Seiten
   hyperfootnotes=false,    % Keine Links auf Fussnoten
   % Bookmarks
   bookmarks=true,          % Erzeugung von Bookmarks fuer PDF-Viewer
   bookmarksopenlevel=1,    % Gliederungstiefe der Bookmarks
   bookmarksopen=true,      % Expandierte Untermenues in Bookmarks
   bookmarksnumbered=true,  % Nummerierung der Bookmarks
   bookmarkstype=toc,       % Art der Verzeichnisses
   % Anchors
   plainpages=false,        % Anchors even on plain pages ?
   pageanchor=true,         % Pages are linkable
   % PDF Informationen
   pdftitle={},             % Titel
   pdfauthor={},            % Autor
   pdfcreator={LaTeX, hyperref, KOMA-Script}, % Ersteller
   %pdfproducer={pdfeTeX 1.10b-2.1} %Produzent
   pdfstartview=FitH,       % Dokument wird Fit Width geaefnet
   pdfpagemode=UseOutlines, % Bookmarks im Viewer anzeigen
   pdfpagelabels=true      % set PDF page labels
]{hyperref}

%% Links auf Gleitumgebungen springen nicht zur Beschriftung, sondern zum Anfang der Gleitumgebung
%  ftp://tug.ctan.org/pub/tex-archive/macros/latex/contrib/oberdiek/hypcap.pdf
\usepackage[all]{hypcap}

%% Erweitert Angabe eines Zitats (im Literaturverzeichnis)
\renewcommand*{\backrefalt}[4]{%
   	% alternative interface
   	% #1: number of distinct back references
   	% #2: backref list with distinct entries
   	% #3: number of back references including duplicates
   	% #4: backref list including duplicates
   	\ifnum#1>0 %
	   	\mbox{(cited on %
	   	\ifnum#1=1 %
			   page~%
		   \else
	   		pages~%
	   	\fi
	   	#2)}%
   	\fi
   }

   
  
